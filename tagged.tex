\section{Geo-Tagged Photos}
In addition to storing data directly in a database stored by a social media
company there is a threat to a user's privacy just through the act of sharing a
image on any website, even those without any social features what so ever.

"Location-based services are rapidly gaining traction in the online world. With
big players such as Google and Yahoo! already heavily invested in the space, it
is not surprising that GPS and WIFI triangulation are becoming standard
functionality for mobile devices: starting with Apple’s iPhone, all the major
smartphone makers are now offering models allowing instantaneously upload of
geo-tagged photos, videos, and even text messages to sites such as Flickr,
YouTube, and Twitter. Likewise, numerous start-ups are basing their business
models on the expectation that users will install applications on their mobile
devices continuously reporting their current location to company
servers.~\cite{friedlcybercasing}"

Data may be stored using the device's camera and GPS functionality. By
combining data into the image including location data you are "Geo-tagging" the
image. Geo-tagging is made possible by using a combination of technologies. The
most prominent of which is GPS which works by triangulating a device's location
using radio signals transmitted from geosynchronous orbiting satellites. 

To store the data the camera applications use a type of file header used within
images. This header is called an Exif header. "EXIF stands for "Exchangeable
Image File Format". This type of information is formatted according to the TIFF
specification, and may be found in JPG, TIFF, PNG, PGF, MIFF, HDP, PSP and XCF
images, as well as many TIFF-based RAW images, and even some AVI and MOV
videos.~\cite{exif}" The data is not limited to location data. It also typically
includes camera model, lens focal length, ISO setting, and even if a flash was
used. 

By including this data into an image and posting the image to the internet the
users are opening themselves up to a threat from the internet. Knowledge of the
threats and the ease of collecting and using this image is one of the main issues
that multimedia users face.

One example of how this can be a threat is directed towards celebrities. "When
Adam Savage, host of the popular science program “MythBusters,” posted a picture
on Twitter of his automobile parked in front of his house, he let his fans know
much more than that he drove a Toyota Land Cruiser. Embedded in the image was a
geo-tag, a bit of data providing the longitude and latitude of where the photo
was taken. Hence, he revealed exactly where he lived. And since the accompanying
text was “Now it’s off to work,” potential thieves knew he would not be at home.
~\cite{geotagnyt}"

\subsection{Accessing Geo-Tagged Information} 
There are several methods for gathering exif data from an image. The first
method is slow and tedious. This method is at the core of reading files. To
access the data the user must open the file using a program called a Hex editor.
it opens binary files using the hexadecimal counting system showing the data in
a form which can be read more easily than in the binary form. While the file is
opened in the hex editor the user can decode the data by converting the hex into
the Unicode characters that are represented. Using a standardized file format
the data can be matched to the way the data is stored. This method is very slow
and takes quite a bit of technical understanding.

The second method is much easier buy is also quite tedious. It requires the
researcher to open each image using a program that exposes the exif data. These
can be small programs specifically meant to show this data or it can be larger
image editing suites such as Adobe's Photoshop. These programs can be used by
anyone with computer knowledge and can be easily downloaded for free from the
internet. Some of these programs do allow for batch reading of images but the
method is not the most effective for searching for specific information.

The main method for analyzing large numbers of images is by using one of the
major photo sharing websites. The sites include Yahoo's Flickr and Google's
Picasa. Both of these sites allow users to search public images based on
location data. Both of these sites also expose this functionality via API. By
using the site's search features or developing tools that use the API
researchers or attackers can learn quite a bit about a person.

\subsection{Privacy Threats} The attacker can search using known location data
or search through a known user's images for location data. By searching using
known location data the attacker is able to learn the user name of a possible
target. This may not seem like much but how often does a person use the same
username or variation on other sites. By combining data collected from all over
the internet a possible attacker is able to assemble a full bio of their target
without having ever met the person.

The Anonymous hacker group uses this technique quite frequently and to great
effect. They call it "doxing" or gaining docs on someone. They have even posted
Doxing is basically finding out information about a person. "The vast majority
of people on the internet are anonymous and with anonymity comes power because
you cannot hurt the physical person behind the username. Doxing takes away this
power, that is why people will go to extreme lengths to make sure they remain
anonymous.  However, the internet stores information from decades ago. Almost
everyone can be found on the internet if you know what to look for. A full
tutorial on one method for gathering information on a person.~\cite{doxing}"

Now this tutorial may not be an authoritative source on researching the internet
for personal information but with this basic tutorial any internet user can gain
some of the skills used by security researchers and can pose a threat to any
one.

In addition to online research and using the meta-data in the images there are
tools that aggregates this location information into one interface. Cree.py is
one such example. "Cree.py is an application that allows you to gather
geo-location related information about users from social networking platforms
and image hosting services. The information is presented in a map inside the
application where all the retrieved data is shown accompanied with relevant
information (i.e. what was posted from that specific location) to provide
context to the presentation.~\cite{creepy}"

Tools and techniques such as Cree.py and doxing may be very real threats but the
main tactic for keeping one's information save is understanding the impact of
posting any type of information on the internet. Millions of users of the
internet and social media applications are opening themselves up to being
exploited and having their identity stolen because they do not have the proper
understanding of how the internet works. Knowledge is one of the best tools
against being exploited and as knowledge grows tools and techniques to protect
one's data will be easier to use and be far more effective.
