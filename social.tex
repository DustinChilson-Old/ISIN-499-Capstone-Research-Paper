\section{Social Media and Personal Data}
%{{{
As discussed earlier, social networking sites are web services that allow users
to have an on-line presence within the confines of the site's network. The question
is what data is collected by these sites and how that data can be used by third
parties as well as the controls that users have over their personal data.

Personal Data is a different subset of data than what people usually associate
with themselves. The typical name, address, phone, etc are just a few points of data
that can be used by different individuals or companies to glean information
about a person. Many social networks collect and use far more data than this as
part of the profiles on their sites. These sites collect data about product
preferences as well as types of content read. How this data is collected and used is
entirely based on the site but in general, social networking sites offer a free
service to users in exchange for showing ads based on a profile that is built
using data that is submitted to the site.

As part of the web2.0 movement many websites added social features to their
sites allowing users to interact with each other as well as the website. Game 
websites added commenting and crowd-sourced reviews about games. On-line
retailers added reviews and allowed users to see what their friends thought of
products. These tools were added to grow interest and increase interaction
between customers and the services offered. More page views usually means more
ad clicks or product sales. As an extension of this many social networking sites
added Application Programming Interfaces (APIs) to their web services allowing
third party programmers to create new and innovative interfaces for their system.
This increased interaction with the service and has grown the service's user base.

APIs are documented "hook in" points for programmers to interact with another
system. To look at it from another view is to see APIs as a computer program
with a window that allows you to control the output of a text box. Using
different switches, buttons, sliders, etc. the user is able to create exactly
the output that they want. For programmers it is useful because they can query
the service for data and receive a string containing the data that they want.

Any reputable website or web application will have a privacy policy and a terms
of use statement which will explain to the site's users what data is collected
and how data collected will be used,
and who the data will be shared with. These documents lay out exactly the rules
with which the site may be used by consumers and how the user's data may be
utilized by the service. These policies play a huge role in the use of
APIs as well. Users may use the APIs to interact with the services but they
limit some of the uses. One limitation may include a "no spam" policy allowing web
services to terminate API access to tools which abuse the API.

Privacy policies, terms of use, APIs, and data collection all play a huge role
in the growth and adoption of social networking platforms. We will look into
several sites data collection practices, API access, and privacy polices to glean some
idea of how some of these sites utilize user data. These sites are divided into two
major sections social networks with geo-location features including Facebook and
Twitter and location dependent social networks such as FourSquare, and Gowalla.%}}}

%==================================================================================
\subsection{Social Networks with Geo-location Features}
%{{{
Some of the first social networks were very basic. They allowed users to create
profiles and interact with one another forming the basic foundation of what
social media is, creating a path of innovation. As these sites grew they added
popular features that were introduced elsewhere, keeping themselves competitive.

Sites that were built early in the social media movement didn't have the benefit
of the research and market data of later social media applications. As other sites grew
they introduced new concepts and experimented with new aspects of how social
media applications function, allowing users to interact in new ways. One such new concept 
that was introduced was the idea of location centric social interactions. One
mechanism which allowed this to flourish was the "Check-in". The "Check-in"
allowed users to post their location as well as a brief message to their
followers.

The first category of social media application are those that have Geo-Location
features but are not dependent on them. These features were added after the
social networking site had already been established and are not central to their
operation.%}}}

%==================================================================================
%  ______              _                 _    
% |  ____|            | |               | |   
% | |__ __ _  ___  ___| |__   ___   ___ | | __
% |  __/ _` |/ __|/ _ \ '_ \ / _ \ / _ \| |/ /
% | | | (_| | (__|  __/ |_) | (_) | (_) |   < 
% |_|  \__,_|\___|\___|_.__/ \___/ \___/|_|\_\
%==================================================================================

\subsubsection{Facebook} 
%{{{
The first example of a site which added geo-location
features after being established is Facebook. Facebook describes itself as
"giving people the power to share and make the world more open and
connected"~\cite{fbabout}. This is a simplistic definition of the largest social
network in the world with over 500 million users and a greater than 15 billion dollar
valuation~\cite{fbcb}. 

Facebook, having the largest user base of all social media applications, has the
unique opportunity to see how a huge subset of the human population interacts. In
addition to being an observer, Facebook can introduce changes into these
interactions. At the core of Facebook is a huge data-store of curated information
about users and by extension humanity as a whole. 

Facebook collects quite a bit of information. "When you sign up for Facebook
you provide [Facebook] with your name, email, gender, and birth date. During the
registration process [Facebook will] give you the opportunity to connect with
your friends, schools, and employers. You will also be able to add a picture of
yourself. In some cases [Facebook] may ask for additional information for
security reasons or to provide specific services to you. Once you register you
can provide other information about yourself by connecting with, for example,
your current city, hometown, family, relationships, networks, activities,
interests, and places. You can also provide personal information about yourself,
such as your political and religious views"~\cite{fbpp}.

The statement of data collected may seem benign enough but it downplays the vast
array of information that is contained within these categories. One example is
the information about relationships and interests. By one user connecting
themselves to another they are implying some shared information. When combining
the interests of both users information can be inferred about both. The more
connections a user has the more concrete the inferred information becomes.

Facebook includes a geo-location feature. This feature is called Facebook Places.
Facebook Places works the same way that a status update works on a user's
profile but also will include a location. This "check-in" feature was developed
to capitalize on the increased usage of smart phones and other mobile devices.

Internally Facebook can access the data on a users profile directly from the
database. The larger the web service the more likely there will be controls in
place to protect the data from external as well as internal threats. To allow
users to access their data using alternative interfaces Facebook provides an API
system to users and developers. Facebook calls their API the Graph API. It
allows the developer to see a user's social graph. The API allows the developer
to access all of the data that is on a profile programmatically. Within the data
their is a user object which associates all of the data to a unique user ID. In
addition to a the basic name, email, gender, and username, developers can access
a user's timezone, bio, birthday, education, hometown, political views, and even
relationship status~\cite{fbapi}. All of these pieces of data are available to
the API developer if a
user gives the appropriate permissions. In addition to having access to a user's
information, the API also gives access to data about that user's friends.

All of this data is returned in a format which is easily readable. Facebook
returns all of the requested information in a format called JSON. "JSON
(JavaScript Object Notation) is a lightweight data-interchange format. It is
easy for humans to read and write. It is easy for machines to parse and
generate. It is based on a subset of the JavaScript Programming Language,
Standard ECMA-262 3rd Edition - December 1999. JSON is a text format that is
completely language independent but uses conventions that are familiar to
programmers of the C-family of languages, including C, C++, C\#, Java,
JavaScript, Perl, Python, and many others. These properties make JSON an ideal
data-interchange language"~\cite{JSON}.

To allow users to protect themselves, Facebook offers a plethora of privacy
controls. These controls are set by the user to change the data that is
available to other people including friends. As another layer of protection
users must explicitly allow information to be given to an application. This
allows the user to protect their information from being gathered easily without
their permission.

Facebook's privacy policy explains that they are certified by TRUSTe, a company
which inspects web sites for security issues, and have met
the requirements for the European Union's Safe Harbor framework. The policy also
explains what data is collected in addition to the information posted to a
profile. This data includes access logs, browser information, as well as network
information such as IP address. The policy finally explains how the Facebook
company will use the data that is collected. They explain that they will use it
to focus ads, make suggestions, and help friends connect with a user.
%}}}

%==================================================================================
%  _______        _ _   _             
% |__   __|      (_) | | |            
%    | |__      ___| |_| |_  ___ _ __ 
%    | |\ \ /\ / / | __| __|/ _ \ '__|
%    | | \ V  V /| | |_| |_|  __/ |   
%    |_|  \_/\_/ |_|\__|\__|\___|_|   
%==================================================================================

\subsubsection{Twitter} 
%{{{
Twitter describes itself as "a real-time information network that connects
you to the latest information about what you find interesting. Simply find
the public streams you find most compelling and follow the conversations. At the
heart of Twitter are small bursts of information called Tweets. Each Tweet is
140 characters in length, but don’t let the small size fool you, you can share a
lot with a little space. Connected to each Tweet is a rich details pane that
provides additional information, deeper context and embedded media. You can tell
your story within your Tweet, or you can think of a Tweet as the headline, and
use the details pane to tell the rest with photos, videos and other media
content"~\cite{twabout}.

Twitter's premise is much different than Facebook's. Where Facebook looks to be
the center of a user's social network, Twitter aims to be a channel for users to
interact with the world. Users follow other users allowing for simple text
message based communication. In most situations Twitter is a one way form of
communication. Popular users, such as celebrities, news networks, and companies,
broadcast information to the population and the users consume this content. An
interesting aspect is that Twitter users can in turn respond to these broadcasts
but are not required to.

Twitter collects very little information upon registration. "When you create or
reconfigure a Twitter account, you provide some personal information, such as
your name, username, password, and email address. Some of this information, for
example, your name and username, is listed publicly on [Twitter's] Services, including
on your profile page and in search results" ~\cite{twpp}.

The bulk of the data that is collected about a user is based on the content that
they share. Simple posts about pizza or bands create an association between a
user's account and some of their interests. The social map of people who a user
follows also tells quite a bit about that user. When all combined into a
comprehensive profile Twitter has a snapshot of a user's interests and can use
it to advertise directly to that user.

By combining this interest profile with location data that can be attached to
Tweets users allow Twitter to focus ads down to a local area. This can provide
the opportunity for
extremely targeted advertising. An example would be a user is interested in pizza, so
Twitter sells advertising space to the local pizza restaurant down the road from the user's
last Tweet with location data. The more a user participates on Twitter the more
accurate the profile becomes and the more targeted the ads become.

Like Facebook, Twitter has an API which returns it's data using the JSON format.
Twitter's API allows for users to access all of the data that is available on a
user's profile. This includes Tweets, re-Tweets, followers, people whom the user
is following, and favorite Tweets.

Twitter's privacy policy explains how Twitter will use the data that is
collected as well as who they will share the data with. Twitter requires that
before they provide any information to a third party the user must consent by
using a form to authorize the transfer of information. Twitter also states that
they may share data with law enforcement if requested. Twitter may also sell
large aggregated data sets which have had personal data removed~\cite{twpp}.

Twitter doesn't seem to gather all that much data about a user from the
perspective of a new registrant but if one digs deeper there is a significant
amount of information to be inferred about individual users that could potentially
be shared.
%}}}

%==================================================================================

\subsection{Location Dependent Networks} 
%{{{
Location dependent social networks are those applications which cannot exist
without users that are willing to share their location data. Facebook and
Twitter have functions that are useful to people who are not comfortable
sharing location data. FourSquare and Gowalla base their entire platform
on mobile users sharing their location data.

Facebook and Twitter are able to monetize their user's interaction and data
without including location data. Location dependent sites on the other hand must
have the location data to create value for the users as well as investors.%}}}

%==================================================================================
%  ______                _____                            
% |  ____|              / ____|                           
% | |__ ___  _   _ _ __| (___   __ _ _   _  __ _ _ __ ___ 
% |  __/ _ \| | | | '__|\___ \ / _` | | | |/ _` | '__/ _ \
% | | | (_) | |_| | |   ____) | (_| | |_| | (_| | | |  __/
% |_|  \___/ \__,_|_|  |_____/ \__, |\__,_|\__,_|_|  \___|
%                                 | |                     
%                                 |_|                     
%==================================================================================

\subsubsection{FourSquare} 
%{{{
FourSquare claims more than eight million users with an additional 35,000 users
being added each day.  Users “check-in” using a smart phone app or text
messaging, proactively reporting their current location.  Users earn points and
badges for check-ins. Frequent check-ins from the same location earn the user
coupons and discounts from the merchants.  

Information that is collected upon registration includes first and last name,
gender, date of birth, email address, telephone number, and current location.
Once an individual is a user of the service, the following information is
collected and stored on servers each time they visit the site: browser or mobile
platform, including user location, IP address, cookie information, and the page
requested~\cite{fspp}. Because this application is dependent on location data
almost all interactions with the site record the user's current location using
their mobile device's GPS or cell phone signal.

As has become a staple of the current batch of social media sites, FourSquare
provides users and developers with an API. This API is a RESTful API. It allows
users to submit requests to a URL and the requested data will be returned. The
data that is returned in the JSON format.

The user has some control over what personal information is shared with others
through the customization of privacy settings. These privacy settings can be
confusing for a new user. To make the task of locking a profile down FourSquare
provides a breakdown of the default settings for different pieces of
information~\cite{fschart}. This chart is available on their website.

FourSquare states that "Personal Information about our users is an integral part
of our business. We neither rent nor sell your Personal Information to anyone"
~\cite{fspp}. Based on this simple statement a user can be reasonably sure that
their data is safe to be submitted to FourSquare. Because of the nature of the
FourSquare service much of the data is already public so users should be aware
of privacy issues related to posting information to the public Internet.%}}}

%==================================================================================
%   _____                    _ _       
%  / ____|                  | | |      
% | |  __  _____      ____ _| | | __ _ 
% | | |_ |/ _ \ \ /\ / / _` | | |/ _` |
% | |__| | (_) \ V  V / (_| | | | (_| |
%  \_____|\___/ \_/\_/ \__,_|_|_|\__,_|
%==================================================================================

\subsubsection{Gowalla} 
%{{{
Gowalla describes it's function as "help[ing] you keep up with friends, share
your favorite places and discover the extraordinary around you"~\cite{goblog}.
Gowalla is a direct competitor to FourSquare, both of the sites operate on the
same use case, allow users to "check-in" to locations giving their friends and
followers access to the locations that they enjoy.

Gowalla collects a set of data that is very close to all of the other sites in
this paper. They state that the site collects a user's name, email, phone number, and
carrier. All of this is standard data to be collected by a mobile based social
media application. Gowalla also collects location information based on the data
passed from the user's mobile device.

Just like Facebook, Twitter, and FourSquare, Gowalla provides an API for
developers and users to connect to the platform and interact with the data that
is available. Gowalla also uses a RESTful API which works the same way as a web
browser functions. Requests are sent to a URL and data is returned. Gowalla
returns their data in JSON format.

Gowalla may share personally identifiable information with third parties.
"Gowalla may share your personally identifiable information with third parties
solely for the purpose of providing services to you. If we do this, such third
parties’ use of your information will be bound by this Privacy
Policy"~\cite{gwpp}.

Gowalla has privacy controls which allow a user to choose who can see what
information from their profile. On Gowalla profiles are called passports and by
default a user's passport is public. By changing a setting a user can change
their passport to private, allowing only their friends to see information.%}}}

%==================================================================================
\subsection{Site Comparison} 
%{{{
Overall these sites are some of the most secure social networking sites on the
Internet today. Security from attack is not the same thing as protecting a
user's privacy. These sites collect a huge amount of information about their
users and make it available to developers.

Facebook and Twitter seem to be the best of the sites for protecting the
information of the users. Both of these sites require that the user give
permission for their data to be shared via their respective APIs. By doing this
control of the data is given back to the users. These sites also state that they will not
sell the data of it's users. They also state that the data may be used for studies to
improve the service offerings but no personally identifying information is
available.

FourSquare is also secure and states that it will not pass data to third
parties. This protects the user's data  as well. Gowalla is one of the sites
that will give information to third parties without consent of the user. This
may be a trusted third party but the fact that the user will not be contacted
gives the user less control over their data.

All of these sites require users to give them information and explain in broad
terms how some of the data will be used. This may be fine with some users but
for others there may be a problem with how much data is available on-line. The
question of privacy is one of choice, is the user willing to use a service which
may compromise their privacy?%}}}

