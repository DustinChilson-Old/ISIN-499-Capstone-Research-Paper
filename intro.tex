\maketitle
Social media (Also known as social networking) is one of the fastest growing forms 
of communication and expression.With the growth of social media as a communication 
tool there have been developments of security and privacy concerns that are just
coming to light. These problems include a breach of privacy to varying levels for 
a person as well as the people that share connections with that person.

To understand the problems associated with social media one must have a firm 
definition of the term social media. There are four aspects of a social 
media service that must be met to qualify as a social media.
\begin{enumerate} 
    \item Web based service.  
    \item Construct a public or semi-public profile within a bounded system.  
    \item Articulate a list of other users with whom they share a connection.  
    \item View and traverse the connections within the system.  
\end{enumerate}
These stipulations create a very specific category of web service to be
considered~\cite{boyd2008social}. Some of the most famous examples of social 
media include Facebook, MySpace, Twitter, and Orkut among hundreds if not 
thousands of others.

Meeting these broad specifications of social media can be quite easy. The
combination with which the features are configured and the rules of the
networking site can create completely different environments.
MySpace for example was a haven for the interaction between bands and their fans
allowing users to share their preferences for music and bands to interact with
their fans. Facebook on the other hand started off as a tool for college
students to connect with their classmates. LinkedIn is even more different as it
focuses on the professional networking that business people do. All three of
these sites are social networks but their implementation and targeted audience
is much different, leading to a different type of user and how those users
interact with the site.

With all of the different implementations of social networks there has been a
growth of social networks that revolve around different aspects of every day
life. One of the growing niche type networks includes location based networks.
These networks have flourished as an increasing number of people have location
aware devices with them at all times. The iPhone for example has a GPS chip
built in which can be used by any application on the device. FourSquare is a
social network that allows users to "check-in" to a location. This is a status
update on the location of the user.

To get a general idea of how popular some of the largest social networking sites
have become just look at the Alexa top 10 list. This is a list of the most
visited websites on the internet. Of the top 10 the first site is Google, the
second is Facebook, and the third is YouTube. Twitter comes in at number nine.
Of the top 10 websites in the world three are social networks.~\cite{alexa}

Some other interesting and telling statistics showing the growth and the amount
of usage some of these sites enjoy include some averages from Facebook's usage
statistics. 
\begin{itemize}
    \item Average user has 130 friends on the site
    \item Average user sends 8 friend requests per month
    \item Average user spends an average 15 hours and 33 minutes on Facebook per month
    \item Average user visits the site 40 times per month
    \item Average user spends an 23 minutes (23:20 to be precise) on each visit
    \item Average user is connected to 80 community pages, groups and events
    \item Average user creates 90 pieces of content each month
    \item 200 million people access Facebook via a mobile device each day
    \item More than 30 billion pieces of content are shared each day
    \item Facebook generates a staggering 770 billion page views per month
\end{itemize}
The possibly most telling two statistic are Facebook hosts more than
629,982,480 users of which 70\% are based outside of the United
States ~\cite{smt} and that including 3D virtual worlds like Second Life and
World of Warcraft there are more social media accounts than there are people in
the world, over 10 billion accounts~\cite{silicon}.

The growth of social media services has brought a host of problems with them.
These problems are not entirely based on the implementation of the sites or the
features that are available at those sites. The problems are based on a lack of
education of the users. The users do not have a functional knowledge of how
the internet works and are unaware of the issues associated with posting
information to a public site.

As children, most kids are taught not to talk to strangers and not to share
personal information on the internet. The advent and growth in popularity of
social networking has changed the thinking of many users. Teens as well as
adults are sharing more and more information to sites that may or may not be
trustworthy or have the user's privacy as a top priority. This is a growing
problem as more and more tools are becoming available to mine the digital
treasure trove that is social media.

Privacy is a relative term. To some people revealing little more than their
name is an invasion and there are others that feel that the information about
them might as well be public. These differing philosophies can make it
challenging for site developers to match features with the level of privacy that
the users want. "We live in a paradoxical world of privacy. On one hand,
teenagers reveal their intimate thoughts and behaviors online and, on the other
hand, government agencies and marketers are collecting personal data about
us."~\cite{barnes2006privacy}
