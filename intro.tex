\maketitle
Social media (Also known as social networking) is one of the fastest growing forms 
of communication and personal expression. With the growth of social media as a communication 
tool, privacy and security concerns not previously considered have come to
light.  These concerns include a breach of privacy to varying levels for 
a person as well as the people that share connections with that person.

To understand the problems associated with social media one must have a firm 
definition of the term social media. Boyd and Ellison state that there are three aspects of a social 
media service that must be met to qualify as a social media. These stipulations create 
a very specific category of web service to be considered a social application.
The social networking site must allow individuals to\ldots
\begin{enumerate} 
    \item Construct a public or semi-public profile within a bounded system.  
    \item Articulate a list of other users with whom they share a connection.  
    \item View and traverse the connections within the
        system~\cite{boyd2008social}.
\end{enumerate}
Some of the most famous examples of social media include Facebook, MySpace, 
Twitter, and Orkut among hundreds if not thousands of others.

Meeting the previously listed criteria can create many sites with the same
features, each
targeting different audiences. MySpace originated as a haven to facilitate the
interaction between bands and their fans; Allowing fans to share their
preferences for music and bands to interact with their fans. Facebook, on the
other hand, started off as a tool for college students to connect with their
classmates. LinkedIn focuses on the professional networking needs of  business
people. All three of these sites are social networks but their implementation
and targeted audience is quite different, leading to a different type of user and
how those users interact with the site.

One of the growing niche type networks includes location based networks.  These
networks have begun to flourish as an increasing number of people have location aware
devices with them at all times. The iPhone for example has a GPS chip built in
which can be used by any application on the device. Facebook and Foursquare are
social networks that allow users to "check-in" to a location using the GPS
technology available in most mobile devices. This is a status update on the
location of the user.

To get a general idea of how popular some of the largest social networking sites
have become look at the Alexa top ten list. This is a list of the most
visited websites on the Internet. Of the top ten the first site is Google, the
second is Facebook, and the third is YouTube. Twitter comes in at number nine.
Of the top ten websites in the world, three are social networks~\cite{alexa}.

Some other interesting and telling statistics showing the growth and usage of
Facebook have been made public. According to Ken Burbary the Facebook... 

\begin{itemize}
    \item Average user has 130 friends on the site.
    \item Average user sends 8 friend requests per month.
    \item Average user spends an average 15 hours and 33 minutes on Facebook per
        month.
    \item Average user visits the site 40 times per month.
    \item Average user spends an 23 minutes (23:20 to be precise) on the site
        each visit.
    \item Average user is connected to 80 community pages, groups and events.
    \item Average user creates 90 pieces of content each month.
    \item 200 million people access Facebook via a mobile device each day.
    \item More than 30 billion pieces of content are shared each day.
    \item Facebook generates a staggering 770 billion page views per
        month~\cite{smt}.
\end{itemize}

Possibly the most telling two statistics
are Facebook hosts more than
629,982,480 users of which 70\% are based outside of the United
States ~\cite{smt} and that including 3D virtual worlds like Second Life and
World of Warcraft there are more social media accounts than there are people in
the world, over 10 billion accounts~\cite{silicon}.

The growth of social media services has brought a host of problems with them.
These problems are not entirely based on the implementation of the sites or the
features that are available at those sites. The problems are based on a lack of
education of the users. The users do not have a functional knowledge of how
the Internet works and are unaware of the issues associated with posting
information to a public site.

Privacy is a relative term. To some people, revealing little more than their
name is an invasion while there are others who feel that all information about
them might as well be public. These differing philosophies can make it
challenging for site developers to match features with the level of privacy that
the users want. "We live in a paradoxical world of privacy. On one hand,
teenagers reveal their intimate thoughts and behaviors on-line and, on the other
hand, government agencies and marketers are collecting personal data about
us\ldots Galkin (1996) states: “Much of the information that people would like
to keep secret is already lawfully in the possession of some company or
government entity, and what we want is to stop further disclosure without
authorization"~\cite{barnes2006privacy}.

As children, most kids are taught not to talk to strangers and not to share
personal information on the Internet. The advent and growth in popularity of
social networking has changed the thinking of many users. Teens as well as
adults are sharing more and more information on sites that may or may not be
trustworthy and may or may not have the user's privacy as a top priority. This is a growing
problem as more and more tools are becoming available to mine the digital
treasure trove that is social media.

This paper will look at some of the most popular social media sites and their
data collection and privacy policies. The paper will look at the the data that is
collected as well as the stated uses for the data. In particular there will be a
focus on the geo-location based features that many applications are adding. The
paper will focus on the utility of using social media applications and will
contrast it with the safety and privacy implications. Finally the paper will
investigate some tools and techniques that are available to protect data on the
Internet.
