\section{Tools for Tracking and Protecting an On-line Presence} 
%{{{
There are tools that are available for users to track what data is available
about them as well as limit the data that is collected on-line. As a great
security practice it is very useful to limit the exposure of personal data to
the world. Just as keeping a close eye on ones credit score, it is important to
keep track of what data is available on the Internet. This section will explain
some of the many tools that are available for users to track the data available
on the Internet as well as introduce some tools that can be used to protect
their personal data.%}}}

\subsection{Tracking Tools} 
%{{{
To keep track of the data that is available on the
Internet there are several options available. The tools can be as simplistic as
visiting every site that there is an account associated with a person and
looking around and a search using Google. The tools can also be extremely
sophisticated and custom built for the purpose of tracking social media
reputation. 

Corporations, elected officials, CEOs, and even private citizens may need to
track the impression of their brand caused by data on the Internet. For this
reason there are custom built services that track the data that is available all
over the Internet and aggregates it into one place for easy analysis. 

Some applications that will scrape all sites for mentions of a user or a set of
keywords includes Radian6 and Trackur. These applications search the scraped
version of the Internet and apply a filter based on the keywords that are
supplied to the application. Radian6 describes itself as a "flexible, web-based
social media monitoring and engagement platform that lets you view relevant
conversations happening around your brand and products in real time. We
aggregate those conversations – saving you lots of leg work – and put them into
visuals that make analysis and measurement meaningful and
actionable"~\cite{Radian6}. This may be more than what is required for the
casual user or job seeker.  There are also open source web applications that
will aggregate all of the posts about, by, and to a user on many of the social
media sites. 

One such project is called ThinkUp. "ThinkUp is a free, open source web
application that captures your posts, tweets, replies, re-tweets, friends,
followers and links on social networks like Twitter and Facebook. With ThinkUp,
you can store your social activity in a database that you control, making it
easy to search, sort, analyze, publish and display activity from your network.
All you need is a web server that can run a PHP application"~\cite{thinkup}.

These tools both work on the same premise, collect information from the Internet
from and about a user's account and create a report for easy
analysis by the user. The differences lay in the reasoning behind them. Radian6
is targeted at brands which are looking to help their customers while
maintaining a specific public image. ThinkUp on the other hand targets
users who would like to have a one stop location of all their data in a database
that is controlled by them where they can interact with their friends safely.
These tools are targeted at different groups but can serve the same purpose of
protecting the information that is available on the Internet about a user.
%}}}

\subsection{Browser Features}
%{{{
As more and more advertisers realize the impact of having extensive profiles on
potential customers, there has been explosive growth in tracking of page views
as well as web searching. Like social media profiles, advertising companies can
gather and infer a plethora of data about a user solely based on the content
that they consume and the topics that they search for. 

To help protect users from tracking and profiles such as those generated by
advertising firms, many of the major browsers have implemented a feature that
allows users to opt out of tracking.  This is based on a recommendation from the
Federal Trade Commission (FTC) in a December 2010 report. "Some consumers are troubled by the collection
and sharing of their information.  Others have no idea that any of this
information collection and sharing is taking place. Still others may be aware of
this collection and use of their personal information, but view it as a
worthwhile trade-off for innovative products and services, convenience, and
personalization. And some consumers, teens for example, may be aware of
the sharing that takes place, but may not appreciate the risks it poses. In
addition, consumers’ level of comfort might depend on the context and amount of
sharing that is occurring"~\cite{ftc}. The FTC goes
on to recommend that the browsers themselves self regulate by implementing
features that would allow consumers to request that their data not be collected.

The Mozilla Firefox browser for example implemented a "Do Not Track" HTTP header
which is sent along with every request to a web server. This explicitly tells
website and ad network operators that the requesting user would prefer not to be
tracked. "When the feature is enabled and users turn it on, web sites will be
told by Firefox that a user would like to opt out of OBA (on-line behavioral
ads). We believe the header-based approach has the potential to be better for
the web in the long run because it is a clearer and more universal opt-out
mechanism than cookies or blacklists"~\cite{fpc}.

This feature brings a great benefit to users but requires that websites voluntarily
adhere to the users request. With no laws being in place to enforce these
features, ad networks may track users anyway.
%}}}

\subsection{Extensions} 
%{{{
In addition to built in browser features many of the current browsers allow for
users to install extensions. These are small add-on pieces of code that extend
the feature set of the browser. These extensions can be small, site specific,
tools that secure the browser. The extensions that help secure
the browser come in many forms and can be used to block tracking cookies, block
sites with known malware, and even rate sites based on parental ratings. These
tools can be used to quickly secure a web browser. They may not protect a user
from sharing too much data on the Internet but will help them stop from sharing data
without their knowledge.

The first type of browser extension is one of the most important. It is the type
of extension that blocks tracking cookies from being used on your computer.
These will also stop some third party services from working on sites. The main
example for this type of extension is called Ghostery. "Ghostery sees the
invisible web - tags, web bugs, pixels and beacons. Ghostery tracks the trackers
and gives you a roll-call of the ad networks, behavioral data providers, web
Ghostery allows you to block scripts from companies that you don't trust, delete
local shared objects, and even block images and iframes"~\cite{ghost}. Ghostery puts your web
privacy back in your hands. Controlling what publishers, and other companies interested in your
activity can see~\cite{ghost}.

Another type of extension is one which can check the reputation of a website.
These tools can be based on a crowd sourced rating or can be based on an
automated security scan. Web of Trust is one such tool which shows a user a
rating of a site. "WOT’s traffic-light style rating system can be understood by
the smallest of web surfers. Green means safe, yellow means caution, and red
means stop"~\cite{wot}.
%}}}

\subsection{Securing Network Connections} 
%{{{
What may be the most overlooked aspect of on-line and social media protective
measures is controlling what information is passed out using a network
connection. As connections are made to the Internet through a browser data is
passed to allow for a smoother transfer. This data includes, browser version,
operating system, screen resolution, window size, extensions, and much more
information. By combining this information together website operators are able
to compile a "browser fingerprint" of a user. This allows the websites to track
a user without setting a cookie.

To protect against these tactics there are a few tools that can help lessen the
impact and control the data that is sent across a network. The first type is a
proxy which, in effect, obscures the network location of the request. A request is
sent to a website to get a website. That request is forwarded to the proper site
and the data is routed to the user. If enough people are using the proxy it
becomes harder to track the individual users because of the network location
mask.

In addition to proxies there are more advanced versions of the same tool. This
is called Tor or The Onion Router, which works like a proxy but instead of the
requests coming from one server the requests come from thousands of different
exit points further obscuring the users data and making it much harder to
fingerprint a user's browser and connection."

The most efficient and safest way to connect to the Internet is through a
secure connection. This can be achieved by "Tunneling" all network traffic
through a Secure Socket Handler (SSH) session or a Virtual Private Network
(VPN) to a server which then makes requests from a secure network server. When
used in combination with Tor or proxies it becomes extremely difficult to track users
who use the proper software to mask their network activity.

These tools are not the end all and be all of network and social media security.
These tools help in securing data leakage but are not a replacement for smart
usage of social media sites. Selective and careful sharing of information is one
to the only ways to be totally secure in the data that is shared on the
Internet.%}}}
