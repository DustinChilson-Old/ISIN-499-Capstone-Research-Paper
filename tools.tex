\section{Tools for Tracking and Protecting an Online Presence} 
There are tools that are available for users to track what data is available
about them as well as limit the data that is collected online. As a great
security practice it is very useful to limit the exposure of personal data to
the world. Just as keeping a close eye on ones credit score it is important to
keep track of what data is available on the internet. This section will explain
some of the many tools that are available for users to track the data available
on the internet as well as introduce some tools that can be used to protect
their personal data.

\subsection{Tracking Tools} To keep track of the data that is available on the
internet there are several options available. The tools can be as simplistic as
visiting every site that there is an account associated with a person and
looking around and a search using Google. The tools can also be extremely
sophisticated and custom built for the purpose of tracking social media
reputation. 

Corporations, elected officials, CEOs, and even private citizens may need to
track the impact of data on the internet. For this reason there are custom built
services that track the data that is available all over the internet and
aggregates it into one place for easy analysis. 

Some applications that will scrape all sites for mentions of include
Radian6 and Trackur These applications search the scraped version of the
internet and applies a filter based on the keywords that are supplied to the
application. Radian6 describes itself as a "flexible, web-based social media
monitoring and engagement platform that lets you view relevant conversations
happening around your brand and products in real time. We aggregate those
conversations – saving you lots of legwork – and put them into visuals that make
analysis and measurement meaningful and actionable ~\cite{Radian6}." This may be
far more than what is required for the casual user or job seeker.  There are
also open source web applications that aggregates all of the posts about, by,
and to a user on many of the social media sites. 

One such project is called ThinkUp. "ThinkUp is a free, open source web
application that captures your posts, tweets, replies, re-tweets, friends,
followers and links on social networks like Twitter and Facebook. With ThinkUp,
you can store your social activity in a database that you control, making it
easy to search, sort, analyze, publish and display activity from your network.
All you need is a web server that can run a PHP application."~\cite{thinkup}

\todo{explain Difference between the tools}

\subsection{Browser Features}
As more and more advertisers realize the impact of having extensive profiles on
potential customers there has been an explosive growth in tracking of page views
as well as web searching. Like social media profiles advertising companies can
gather and infer a plethora of data about a user solely based on the content
that they consume and the topics that they search for. 

To help protect users from tracking and profiles such as these many of the major
browsers have implemented a feature that allows users to opt out of tracking.
This is based on a recommendation from the FTC in a December 2010 report. "Some
consumers are troubled by the collection and sharing of their information.
Others have no idea that any of this information collection and sharing is
taking place. Still others may be aware of this collection and use of their
personal information but view it as a worthwhile trade-off for innovative
products and services, convenience, and personalization. And some consumers –
some teens for example – may be aware of the sharing that takes place, but may
not appreciate the risks it poses. In addition, consumers’ level of comfort
might depend on the context and amount of sharing that is occurring.~\cite{ftc}"
The FTC goes on to recommend that the browsers themselves self regulate by
implementing features that would allow consumers to request that their data not
be collected.

The Mozilla Firefox browser for example implemented a "Do Not Track" HTTP header
which is sent along with every request to a web server. This explicitly tells
website and ad network operators that the requesting user would prefer not to be
tracked. “When the feature is enabled and users turn it on, web sites will be
told by Firefox that a user would like to opt out of OBA. We believe the
header-based approach has the potential to be better for the web in the long run
because it is a clearer and more universal opt-out mechanism than cookies or
blacklists.~\cite{fpc}”

This feature is a great asset to users but requires that websites voluntarily
adhere to the users request. With no laws being in place to enforce these
features ad networks may track users anyway.

\subsection{Extensions} 

\todo{AdBlock}

\todo{AdBlocking} 

\todo{Ghostery}

\todo{etc.}

\subsection{Anonimization} 

\todo{proxies}

\todo{VPN}

\todo{Tor}

\subsection{Script Kiddie Toys}

\todo{Cree.py}
